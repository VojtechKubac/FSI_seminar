\section{Turek\&Hron benchmark}
In this section we describe in detail the Turek\&Hron benchmark, it more or less follows the 
layout of the original paper. In the first part we provide 
the setting of the benchmark, like the governing equations, interaction conditions, 
boundary and initial conditions and the considered geometry.

Then two parts about partition tests follow. The first part is about the pure elasticity problem,
wheres the second one deals with the pure fluid problem. The last part of this section describes
the FSI benchmarks.


\subsection{The Benchmark Setting}
We consider the flow of an \emph{Incompressible Newtonian} homogeneous fluid around a rigid circle with
attached elastic flap. We denote the fluid domain as $\Omega_t^f$ and the domain of elastic solid
as $\Omega_t^s$, where the subscript $t$ denotes time since we consider changing geometries. 
Next, we define the interface between the fluid and elastic solid as 
$\Gamma_t = \partial\Omega_t^f \cap \partial\Omega_s^t$.
Note that the rigid circle is not considered as part of computational domain.
\paragraph{Governing Equations} 
As said above, the fuid is considered to be incompressible Newtonian and homogeneous. Its state is
therefore described by the velocity $\mathbf{v}^f$ and pressure $p^f$. The governing equations are
\begin{equation}
  \begin{aligned}
    \rho^f \partial_t \mathbf{v}^f + \rho^f(\vf\cdot\nabla)\vf &= \mathrm{div}\,\mathbb{T}^f \\
    \mathrm{div}\,\mathbf{v}_f &= 0
  \end{aligned}\quad
  \mathrm{in}\, \Omega_t^f.
\end{equation}
The symbol $\rho^f$ denotes the constant fluid density and $\mathbb{T}^f$ denotes the 
\emph{Cauchy stress tensor} which is given by the constitutive relation
\begin{equation}
  \mathbb{T}^f = -p^f\mathbb{I} + \rho^f\nu^f(\nabla\vf + {\nabla\vf}^T),
\end{equation}
where $\mu^f$ is the constant viscosity of the fluid.

Let us now describe the properties of the solid structure. The structure is assumed to be 
\emph{elastic} and \emph{compressible}. The standard way to write the balance equation for solid is
using the Lagrangian description, with respect to some fixed (usually initial) configuration 
$\Omega^s$,
\begin{equation}
  \hat{\rho}^s \partial_{tt} \usref = \hat{\rho}_s\hat{\mathbf{g}_s} 
  + \widehat{\mathrm{div}}\,\hat{\mathbb{T}}^{{(1)}^s},
\end{equation}
where $\hat{\rho}^s$ is the refferential density, $\hat{\mathbf{g}_s}$ is the volume force
acting on the body and 
$\hat{\mathbb{T}}^{{(1)}^s} = (\mathrm{det} \mathbb{F})\mathbb{T}\mathbb{F}^{-T}$
is the \emph{1st Piola-Kirchhoff stress tensor}. Finaly $\mathbb{F} = \mathbb{I} = \nabla \usref$
is the deformation gradient. For more detail see any introdactory book to continuum mechanics.

The material is specified by the constitutive law, in the benchmark 
\emph{St. Vennant-Kirchhoff} model was used. The standard way to write the equation is with 
the use of the \emph{2nd Piola-Kirchhoff stress tensor} 
$\hat{\mathbb{T}}^{{(2)}^s} = \mathbb{F}^{-1}\hat{\mathbb{T}}^{{(1)}^s}
 = (\mathrm{det} \mathbb{F})\mathbb{F}^{-1}\mathbb{T}\mathbb{F}^{-T}$.
The contitutive law is
\begin{equation}
  \hat{\mathbb{T}}^{{(2)}^s} = \lambda_s Tr(\hat{\mathbb{E}}_s)\mathbb{I} + 2 \mu_s\hat{\mathbb{E}}_s,
\end{equation}
where $\mathbb{E} = \frac{1}{2} \left( \mathbb{F}_s^T\mathbb{F}_s - \mathbb{I} \right)$
and $\lambda_s$ and $\mu_s$ are \emph{Lamé constants}.

An alternative pair of constants constists of the \emph{Poisson ratio} $\mu^s$ 
($\mu^s < 0 $ for a compressible structure) and the \emph{Young modulus} $E$, these are
easier to measure. The relation between these to sets is done by
\begin{align}
  \nu^s &= \frac{\lambda^s}{2(\lambda^s + \mu^s)} \quad
  &E &= \frac{\mu^s(3\lambda^s + 2\mu^s)}{(\lambda^s + \mu^s)} \\
  \mu^s &= \frac{E}{2(1+\nu^s)} \quad
  &\lambda^s &= \frac{\nu^sE}{(1+\nu^s)(1-2\nu^s)}.
\end{align}


\paragraph{Interaction Conditions}
As stated above, we consider a viscous Newtonian fluid, for which usually the \emph{no-slip} 
boundary condition is used on the boundary with solid structure. This condition means, that
the fluid sticks to the solid and has therefore the same velocity as the structure.
Very often the structure does not move (in our case the walls of the channel and surface
of rigid circle) and this condition is then just a zero Dirichlet boundary. This is however
not the case of the FSI interface, here the no-slip condition reads
\begin{equation}
\vf(x,\,t) = \vsref(\chi_s^{-1}(x,\,t),\,t) \quad\mathrm{on}\;\Gamma^t.
\end{equation}
Or, using Lagrangian variables
\begin{equation}
\vf(X + \usref(X,\,t),\,t) = \vsref(X,\,t) \quad \mathrm{on}\;\Gamma^t.
\end{equation}
This condition is sometimes called the \emph{kinematic condition}.
There is one more condition we would like to be satisfied, this is an application of Newton's
action-reaction law. More specifically, we require that the forces on the interface to be in balance,
\begin{equation}
  \mathbb{T}_f\mathbf{n} = \mathbb{T}_s\mathbf{n}\quad\mathrm{on}\;\Gamma^t.
\end{equation}
This condition is known under the name \emph{dynamic condition}.



\paragraph{Boundary and Initial Conditions}
As indicated above, we assume that the fluid sticks on the walls and on the surface of the rigid 
circle, so we imply the zero Dirichlet boundary condition for fluid velocity $vf$ there. 
We also assume that the elastic flap is fixed to the rigid circle, so we again assume the 
zero Dirichlet boundary condition, this time for the solid displacement $\usref$.
On the left boundary we presibe a parabolic inflow 
\begin{equation}\label{eq:steady_profile}
  \vf(0, \,y) = 1.5\bar{U}\frac{y(H-y)}{(\frac{H}{2})^2},
\end{equation}
where $\bar{U}$ is the mean flow velocity and $H$ is the height of the channel.
The maximum velocity of the profile is just $1.5\bar{U}$.
On the right boundary we set the
\emph{do-nothing} boundary condition, $\mathbb{T}^f\mathbf{n} = \mathbf{0}$.

Because of computational reasons, it is reasonable to start the time dependent tests from zero
velocity and zero displacement and then start increasing the inflow boundary condition. 
The approach suggested in the paper is
\begin{equation}
  \vf(t,\,0,\,y) = \begin{cases} \vf(0,\,y)\frac{1 - \mathrm{cos}(\frac{\pi}{2}t)}{2}
     \quad &\mathrm{if}\, t < 2.0 \\
  \vf(0,\,y) \quad &\mathrm{otherwise},
  \end{cases}
\end{equation}
where $\vf(0,\,y)$ is the velocity profile from (\ref{eq:steady_profile}).


\paragraph{Geometry}
The domain is based on the classical 2D benchmark flow around the cylinder \cite{TurekSchaefer1996}.
The only difference is that we consider an elastic flap stick to the rigid cylinder, (and the
length of the channel is little bit higher).
So the geometry is given by
\begin{itemize}
  \item The domain dimensions are: length L = 2.5, height H = 0.41.
  \item The circle center is positioned at C = (0.2, 0.2) (measured from the left
bottom corner of the channel) and the radius is r = 0.05.
  \item  The elastic structure bar has length l = 0.35 and height h = 0.02, the
right bottom corner is positioned at (0.6, 0.19), and the left end is fully
attached to the fixed cylinder.
  \item The control points are A(t), fixed with the structure with A(0) = (0.6, 0.2),
and B = (0.15, 0.2).
\end{itemize}


\subsection{Quantities for Comprarision}
For comparison of results three different quantities will be used
\begin{enumerate}
  \item The displacement at the end of the elastic beam, at the point $A(t)$.

  \item Forces exerted by the fluid on the whole submerged body, i.e. lift and
drag forces acting on the cylinder and the beam structure together 
    \begin{equation}\label{eq:forces}
      (F_D, F_L) = \int_S\,\mathbb{T}\mathbf{n}\,d\sigma,
    \end{equation}
    where $S = S_1 \cup S_2$ (see Fig....) denotes the part of the circle being in
contact with the fluid (i.e. $S_1$) plus part of the boundary of the beam
    structure being in contact with the fluid (i.e. $S_2$), and $\mathbf{n}$ is the outer unit
normal vector to the integration path with respect to the fluid domain.

  \item Pressure difference between points $A(t)$ and $B$
    \begin{equation}\label{eq:press_diff}
      \Delta p (t) = p(B) - p(A(t)).
    \end{equation}

\end{enumerate}



\subsection{The Fluid Problem}
The first validation problem is the fluid problem. These tests are refered to as CFD tests 
(the abbreviation states for Computational Fluid Dynamisc). Here the geometry is the same as in the
full setting, with the difference that the flap behind cylinder is completely rigid.
There are two possibilities for that, the first, more srtaightforward one, is to consider just the
fluid domain $\Omega_t^f$ (which is now independent of time) and prescibe standard no-slip condition
on the interaction between the fluid and the flap. The other possibility is to model
full FSI setting while making the solid almost rigid by setting large structural parameters
($\rho^s = 10^6 \frac{\mathrm{kg}}{\mathrm{m}^3}$, $\mu^s = 10^{12} \frac{\mathrm{kg}}{\mathrm{ms}^2}$).
The first choice is more than natural for partitioned approach, while for the 
monolithical approach it requires generating of a new mesh and huge changes in the code.

Three different choices for the problem specific constants are considered (see the Table 
\ref{table:CFD_constants}), the first two
leads to a stationary solution, wheres the third results in turbulatory flow. Values for comparison
are the forces exerted by the fluid.

\begin{table}\label{table:CFD_constants}
\centering
\begin{tabular}{|l|r|r|r|}
  \hline
  dimensional parameter                               & CFD1 & CFD2 & CFD3 \\
  \hline
  $\rho^f\,[10^3 \frac{\mathrm{kg}}{\mathrm{m}^3}]$   &    1 &    1 &    1 \\
  $\nu^f\,[10^{-3} \frac{\mathrm{m}^2}{\mathrm{ms}}]$ &    1 &    1 &    1 \\
  \hline
  $\bar{U}$                                           & 0.2  &    1 &    2 \\
  \hline \hline
  non-dimensional parameter                           & CFD1 & CFD2 & CFD3 \\
  \hline
  $\mathrm{Re} = \frac{\bar{U}d}{\nu^f}$              &   20 &  100 &  200 \\
  $\bar{U}$                                           & 0.2  &    1 &    2 \\
  \hline
\end{tabular}
  \caption{Parameter settings for the CFD tests}
\end{table}


\subsection{The Solid Problem}
The elasticity validation problem consideres a the elastic flap on which acts a volume
(gravitational) force $\mathbf{g} = (0, \,g ) [\frac{\mathrm{m}}{\mathrm{s}^2}]$.
The test problems are abbrevated as CSM problems (Computational Structural Mechanics).
The set of the test problems again consists of diferent choices of problem specific paramaeters
(prescribed in table \ref{table:CSM_values}).
Another difference is that, the first two problems, CSM1 and CSM2, are computed as steady
state solutions, while the CSM3 problem is a time dependent solution, where we start from
undeformed configuration and at the time $t=0$ the volume force starts to act.
This results in oscilatorial movement of the flap, since we assume no dissipation or resistance
there is no damping and one should obtain perfectly elastic behaviour.

\begin{table}\label{table:CSM_constants}
\centering
\begin{tabular}{|l|r|r|r|}
  \hline
  dimensional parameter                               & CSM1 & CSM2 & CSM3 \\
  \hline
  $\rho^s\,[10^3 \frac{\mathrm{kg}}{\mathrm{m}^3}]$   &    1 &    1 &    1 \\
  $\nu^s$                                             &  0.4 &  0.4 &  0.4 \\
  $\mu^f\,[10^{6} \frac{\mathrm{kg}}{\mathrm{ms}^2}]$ &  0.5 &  2.0 &  0.5 \\
  \hline
  $g$                                                 &    2 &    2 &    2 \\
  \hline \hline
  non-dimensional parameter                           & CSM1 & CSM2 & CSM3 \\
  \hline
  $\nu^s$                                             &  0.4 &  0.4 &  0.4 \\
  $E^s$                   & $1.4\cross10^6$ &  $5.6\cross10^6$ &  $1.4\cross10^6$ \\
  \hline
  $g$                                                 &    2 &    2 &    2 \\
  \hline
\end{tabular}
  \caption{Parameter settings for the CSM tests}
\end{table}



\subsection{The Fluid-Structure Interaction}
Now we finaly get to the Fluid-Structure Interaction problems, so the abbreviation of these final 
tests is FSI. Three cases are considered again, the first one (FSI1) results in a steady solution,
while the other two later (FSI2, FSI3) results in turbulent  flow and oscilating flap.
FSI3 has faster oscilations and is a little more challenging to obtain the right results, 
compared to FSI2.
Note that the fluid parameters for FSI problems correspond to the parameters from CFD problems.

\begin{table}\label{table:CSM_constants}
\centering
\begin{tabular}{|l|r|r|r|}
  \hline
  dimensional parameter                               & FSI1 & FSI2 & FSI3 \\
  \hline
  $\rho^s\,[10^3 \frac{\mathrm{kg}}{\mathrm{m}^3}]$   &    1 &   10 &    1 \\
  $\nu^s$                                             &  0.4 &  0.4 &  0.4 \\
  $\mu^f\,[10^{6} \frac{\mathrm{kg}}{\mathrm{ms}^2}]$ &  0.5 &  0.5 &  2.0 \\
  \hline
  $\rho^f\,[10^3 \frac{\mathrm{kg}}{\mathrm{m}^3}]$   &    1 &    1 &    1 \\
  $\nu^f\,[10^{-3} \frac{\mathrm{m}^2}{\mathrm{ms}}]$ &    1 &    1 &    1 \\
  \hline
  $\bar{U}$                                           & 0.2  &    1 &    2 \\
  \hline \hline
  non-dimensional parameter                           & FSI1 & FSI2 & FSI3 \\
  \hline
  $\beta = \frac{\rho^s}{\rho^f}$                     &    1 &   10 &    1 \\
  $\nu^s$                                             &  0.4 &  0.4 &  0.4 \\
  $\mathrm{Ae} = \frac{E^s}{\rho^f\bar{U}^2}$ & $3.5\cross10^4$ &  $1.3\cross10^3$ &  $1.4\cross10^3$ \\
  \hline
  $\mathrm{Re} = \frac{\bar{U}d}{\nu^f}$              &   20 &  100 &  200 \\
  $\bar{U}$                                           & 0.2  &    1 &    2 \\
  \hline
\end{tabular}
  \caption{Parameter settings for the FSI tests}
\end{table}


\subsection{Further Publications}
The test problems described above are computed by several researcher groups using different 
approaches, the results of their computations are presented in \cite{TurekHronRazzaqSchaefer2010}.
There is also a webpage \cite{featflow} with the benchmark setting and 
the reference values can be downloaded there.




\begin{figure}[h]%
 	\begin{center}%
 		\includegraphics[scale=0.1]{figure1.png}%
 		\caption{Tree}\label{fig:baum}%
 	\end{center}%
\end{figure}

\begin{table}[h]%
 	\begin{center}%
		\caption{Some table}\label{tab:example}%
	 	\begin{tabular}{c|c}%
 			Column1 & Column2\\
 			\hline
 			0 & 1\\
 		\end{tabular}%
 	\end{center}%
\end{table}
